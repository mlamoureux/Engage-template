%  Template for NSERC Engage Grant
% M. Lamoureux 2016
% Created for PIMS

\documentclass[12pt]{article}

% We set the margins and headers to satisfy NSRERC Requirements

\usepackage{fancyhdr}

\setlength{\textheight}{9.0in}
\setlength{\topmargin}{-.5in}
\setlength{\textwidth}{7in}
\setlength{\oddsidemargin}{-.25in}
\setlength{\evensidemargin}{-.25in}


\pagestyle{fancyplain}
\rhead[]{}
\lhead[]{\sc Engage Grant \hfill Contributions from Organization \hfill  PIN 112358: Lastname, Initials}  % Make sure you add your real name, initials, and PIN
\chead[]{}
\begin{document}

\setcounter{page}{5}
\subsection*{Contributions from Industrial Partner}

In this one page, indicate both the cash and in-kind support that the company is providing for the internship. Explain the need for the contribution. Note that the partner company needs to provide a letter of support acknowledging their commitment to supply these contributions, so be sure to discuss it with the company before putting the information into the grant application. 

A sample itemization might look like this:

\vskip 3mm \noindent
Partner company's contributions (cash and in-kind) are as follows:
\begin{itemize}
\item Salary top-up for intern (cash): \$10,000 .
\item Purchase of equipment. \$2,000.
\item Research supervisor at Company:  10 hours/week for 20 weeks, at \$50/hr is \$10,000.
\item Technical staff. 5 hours/week for 20 weeks at \$50/hr is \$5,000.
\item Office space for 20 weeks. \$2,000.
\end{itemize}

Total: \$12,000 in cash contribution, \$17,000 in-kind contribution.

\vskip 3mm

 \end{document}
 \end

