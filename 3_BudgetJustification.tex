%  Template for NSERC Engage Grant
% M. Lamoureux 2016
% Created for PIMS

\documentclass[12pt]{article}

\usepackage{fancyhdr}

% We set the margins and headers to satisfy NSRERC Requirements

\setlength{\textheight}{9.0in}
\setlength{\topmargin}{-.5in}
\setlength{\textwidth}{7in}
\setlength{\oddsidemargin}{-.25in}
\setlength{\evensidemargin}{-.25in}


\pagestyle{fancyplain}
\rhead[]{}
\lhead[]{\sc Engage Grant \hfill Budget Justification \hfill  PIN 112358: Lastname, Initials}  % Make sure you add your real name, initials, and PIN
\chead[]{}


\begin{document}

\setcounter{page}{6}
%\maketitle

%\section{Most Significant Contributions to Research and/or to Practical Applications}
\subsection*{Budget Justification}

This section gives a written description of what is needed in the budget. There is another section on the NSERC form where you put in a list of items for salary, equipment, travel, publication costs, and anything else that is relevant. With a total listed. 

So on this page, explain in sentences how the numbers in your budget translates to what you are doing. For instance, you could say you will spend \$30,000 for salary of the intern, who is a PhD candidate, being paid at a rate of \$30/hour. This could include \$10,000 cash contribution from the company, and \$20,000 from the grant. (But note that the company is not required to make a cash contribution.) Benefits are calculated in addition to this (give the percentage rate for benefits at your university). If you need specialized equipment or access to data sources that charge a user fee, indicate it here and explain why you need it. Similarly for all other expense,

Indicate the total cost. Note that NSERC provides up to \$25,000 for one internship. Total costs can be higher, if the industrial partner is adding funds to the project. Indicate that all here. 


\vskip 3mm \noindent
You have one page for the budget justification. 



 \end{document}
 \end

